\documentclass[write-up.tex]{subfiles}
\begin{document}

\section{Introduction}
The field of simulation is one with many applications in all industries, with much overlap between Mathematics, Physics and Computer Science due to its predictable behaviour. One such application is Computational Fluid Dynamics (CFD), or in other words predicting the movement of fluids, which will be the focus for this project.

Simulating fluids involves observation of fluid phenomena such as wind, weather, ocean waves, waves induced by ships or simply pouring a glass of water. Such phenomena may seem extremely trivial at first glance, but in reality involve a deeper understanding of physical, mathematical and algorithmic methods.


\subsection{Motivation}

My motivation for this project stems from the work of Sebastian Lague \cite{Lague}, a graphics developer who shares his exemplar work on \href{https://github.com/}{Github} and through digital media on \href{https://youtube.com}{YouTube}. Through his work, I was introduced to the concept of Smoothed Particle Hydrodynamics in the Computer Graphics community and was given great insight into its implementation in industry. Further reading, especially into the sources of Lague, piqued my interest and only reinforced the idea of undertaking this concept because it provided the overlap between Mathematics, Physics and Computer Science, it was far beyond the scope of the A level curriculum but most importantly it provided a means to challenge, extend and implement new knowledge in a field which I plan to undertake in the future.

\subsection{Definitions}
In this section, I provide clear and concise definitions for the key terms essential to understanding the context and methods presented in this paper.
\begin{worddefs}
\item[Acceleration.] The rate of change of velocity.
\item[Advection.] The horizontal movement of a mass of fluid.
\item [Algorithm.] A set of instructions used to solve a particular problem or perform a specific task.
\item[Brownian Motion.] The random motion of particles suspended in a medium.
\item[Compute time.] The time required for a computer system to perform a specific task or calculation.
 \item [Computer Graphics.] A technology that generates images and videos on a computer screen, also referred to as CG.
\item[CPU.] The Central Processing Unit (CPU) is the primary component of a computer that acts as its “control center".
\item [Debug.] The process of finding and fixing errors (bugs) in source code.
\item[Density.] The compactness of a substance, or the mass per unit volume.
\item[Eulerian.] A grid based approach to simulation.
 \item [Fluid.] Any substance which flows due to applied forces, namely liquids and gasses.
 \item[Force.] An influence which causes an object to accelerate.
\item [Frame.] A single image which makes up a collection of images for an animation.
\item[Friction.] A force resisting the relative motion of an object.
\item[GPU.] The Graphics Processing Unit (GPU) is a specialized electronic circuit  designed for digital image processing and to accelerate computer graphics calculations.
\item[Intermolecular Forces.] The attractive or repulsive forces which arise between the molecules of a substance.
\item[Kinetic energy.] The energy an object possesses due to its motion.
\item [Lagrangian.] A particle based approach to simulation.
\item [Liquid.] A type of fluid which takes the shape of any container or vessel it is stored within.
\item[Mass.] The measure of the amount of matter in a system.
\item [Optimisation.] Modifying an algorithm or software to reduce the usage of computer resources or compute time.
\item[Pressure.] The physical force exerted on an object by something in contact with it, or the force per unit area.
\item [Pseudocode.] Writing an algorithm in plain English for design purposes.
\item[Render.] The process of generating a photorealistic or non-photorealistic image from a 2D or 3D model.
 \item [Simulation.] Imitation of a situation or process.
\item[Surface Tension.] The tension on the surface of a liquid caused by the attraction of particles in the surface layer, tends to minimise surface area.
 \item[Velocity.] Speed of an entity associated with a direction.
 \item[Viscosity.] A quantity defining the magnitude of the internal friction in a fluid, or the Pressure resisting uniform flow.
\end{worddefs}

\subsection{Smoothed Particle Hydrodynamics}
Smoothed Particle Hydrodynamics (SPH) stands out as a Lagrangian approach to fluid simulation, offering a dynamic method for modeling complex fluid behavior. Developed in 1977 from the work of Lucy \cite{lucy} and Gingold and Monaghan \cite{gingold} in astrophysics, it posed as a strong alternative to existing methods at the time. Its transformative potential was further realized in interactive liquid simulation, thanks to the efforts of Müller \textit{et al.} \cite{muller} in 2003.

In SPH, the spatial domain is approximated into particles, each embodying various fluid properties like mass, density, and velocity. Throughout the simulation, these particles dynamically interact, forming a fluid-like continuum. Notably, the field quantities characterizing the fluid, such as pressure or velocity, can be precisely evaluated at any point in space by observing the overlapping influence spheres of individual particles. Adaptability and precision makes SPH a compelling choice for simulating fluid phenomena across a spectrum of scales and applications.

\subsection{Outline and Structure}
I plan to code a semi-realistic 2-D animation of a fluid in the programming language C++. This will involve describing fluid phenomena mathematically to come up with a theoretical model. I will then implement each section of the theoretical model, test its efficacy and possibly look into optimisation techniques as required. Finally to evaluate the success of my simulation I will check against the success criteria, reverting to previous methods of development if necessary.
%justify the use of C++

\subsection{Success Criteria}
The success criteria is as follows:
\begin{itemize}
 \item Research SPH and popular alternative methods of fluid simulation.
 \item Develop a simulation which solves the Navier Stokes Equations and therefore can be classed as a fluid.
 \item Have an animation that runs with minimal lag and resource wastage.
 \item Add window resizing, mouse forces and sliders for user interaction.
 \item Ensure the simulation can respond to user interaction.
\end{itemize}
\subsection{Skills}
 \begin{itemize}
  \item Read and understand academic publishings and research papers in a Mathematics and/or Computer Science background.
  \item Be able to take key information and synthesize it to create my own model.
  \item Typeset my written work in LaTeX, a document formatting tool.
  \item Learn to use a new programming language, C++.
  \item Learn to add interactive elements to projects in C++ using SFML and ImGui.
  \item Test and debug code in C++.
  \item Manage my source code using version control software, Git/Hub.
  \item Manage my time and changes in timescales.
 \end{itemize}
 \clearpage
\end{document}
