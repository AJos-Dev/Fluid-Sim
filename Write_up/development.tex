 \documentclass[write-up.tex]{subfiles}
 \begin{document}
\section{Development}
\subsection{Boilerplate code}

The beginning of the implementation involved the boilerplate minimal code required for my graphics library SFML to be setup as well as including key libraries which will be needed throughout the project.
\begin{lstlisting}

#include <SFML/System/Vector2.hpp>
#include <SFML/Graphics.hpp>
#include<iostream>
#include <algorithm>
#include<vector>
#include<cmath>
#include <omp.h>

int main()
{
    //Initialize SFML
    sf::RenderWindow window(sf::VideoMode(900, 900), "Smoothed Particle Hydrodynamics Simulation");
    window.setFramerateLimit(120);
    sf::View view = window.getDefaultView();
    sf::Vector2u window_size = window.getSize();

    while (window.isOpen())
    {
        sf::Event event;
        while (window.pollEvent(event))
        {
            if (event.type == sf::Event::Closed)
                window.close();
            if (event.type == sf::Event::Resized){
                sf::FloatRect visibleArea(0.f, 0.f, event.size.width, event.size.height);
                window.setView(sf::View(
                visibleArea));
            }
        }
        sf::Vector2u window_size = window.getSize();
        window.clear();
        window.display();
    }
    return 0;
}
\end{lstlisting}

\subsection{Particle initialization}
Using instances of a Particle struct, which stores the information each particle needs, particles can start being added. This includes position and velocity vectors, the circular shape, local densities and pressures and their predicted position.

\begin{lstlisting}
struct particle{
    sf::CircleShape droplet{particle_radius};
    sf::Vector2f position{0.f, 0.f};
    sf::Vector2f velocity{0.f, 0.f};
    float local_density = 0.f;
    float local_pressure = 0.f;
    sf::Vector2f predicted_position{0.f, 0.f};
};
\end{lstlisting}

A dynamic array of these particles is initialized and placed in an orderly fashion at the start of the simulation. Their colour is also set to cyan for now. Refer to Figure \ref{fig:cyan} in Appendix \ref{appendix:a}.

\begin{lstlisting}
const int particle_num = 1600;
const float particle_mass = 1.f;
vector<particle> particles(particle_num);

void placeParticles(){
    int particlesPerRow = sqrt(particle_num);
    int particlesPerColumn = (particle_num - 1)/particlesPerRow + 1;
    int spacing = 2.5*particle_radius;
    for (int i = 0; i < particle_num; i++){
        particles[i].position.x = (i % particlesPerRow + particlesPerRow / 2.5f +0.5f) * spacing;
        particles[i].position.y = (i / particlesPerRow + particlesPerColumn / 2.5f +0.5f) * spacing;
        particles[i].droplet
        .setFillColor(sf::Color::Cyan);
    }
}

\end{lstlisting}

\subsection{Simulation loop}
In order to calculate new positions of particles, they must be iterated over every step of the simulation to calculate new positions using current velocity and acceleration vectors and multiplying them by $dt$, the change in time per simulation step, and finally drawn to screen. These vectors can be updated by implementing gravity first.
\begin{lstlisting}
const float gravity = 9.81f;
const float dt = 1.f/60.f;

 void resolveGravity(int i){
    particles[i].velocity.y += gravity * dt
}
\end{lstlisting}

And within the simulation loop:
\begin{lstlisting}
for (int i = 0; i < particle_num; i++){
    resolveGravity(i);
    particles[i].position.x += particles[i].velocity.x * dt;
    particles[i].position.y += particles[i].velocity.y * dt;
    particles[i].droplet
    .setPosition(particles[i]
    .position.x-particle_radius,
    particles[i].position.y
    -particle_radius);

    //Draw particles
    window.draw(particles[i].droplet);
\end{lstlisting}
Appendix \ref{appendix:a} Figure \ref{fig:nograv} (\href{https://youtube.com/shorts/FSuH_Cs1Qh4?feature=share}{video}) shows the implementation of gravity but with no simulation border. This can be implmented with ease in the following manner:

\begin{lstlisting}
void resolveCollisions(int i, sf::Vector2u window_size){
    if (particles[i].position.x > window_size.x || particles[i].position.x < 0){
        particles[i].position.x = clamp((int)particles[i].position.x, 0, (int)window_size.x);
        particles[i].velocity.x *= -1 * collision_damping;
    }
    if (particles[i].position.y >= window_size.y || particles[i].position.y <= 0){
        particles[i].position.y = clamp((int)particles[i].position.y, 0, (int)window_size.y);
        particles[i].velocity.y *= -1 * collision_damping;
    }
}
\end{lstlisting}

A collision damping factor has been implemented here as well to simulate the loss the energy upon collision with the simulation border. The user can also successfully interact with the simulation by resizing the window, hitting one of the sucess criteria for this project. Refer to Appendix \ref{appendix:a} Figure \ref{fig:grav} (\href{https://youtube.com/shorts/wO0xUBSKVck?feature=share}{video}).
\subsection{Predicted Position Optimisation}

The predicted position of the particles, which will be used throughout the development, need to be calculated first. This is done quite easily in this manner:

\begin{lstlisting}
 void predictPositions(int i){
    particles[i].predicted_position.x = particles[i].position.x + particles[i].velocity.x * dt;
    particles[i].predicted_position.y = particles[i].position.y + particles[i].velocity.y * dt;
}
\end{lstlisting}
The function simply predicts the position of the particle depending on the current velocity.


\subsection{Smoothing Kernel and derivative}

As per the theoretical model, the preferred choice of smoothing kernel is the customised version of $W_{\text{spiky}}(r, h)$, allowing for large repulsion forces at small distances and less computationally strenuous powers. In C++, this looks like the following:

\begin{lstlisting}
float smoothingKernel(double dst){
    float volume = pi * (float)(pow(smoothing_radius, 4.0)/2);
    if (dst >=smoothing_radius){
        return 0.f;
    }
    float value = (float)pow(smoothing_radius - dst, 3.0);
    return value/volume;
}
\end{lstlisting}

Notice how ``0'' is returned if the distance is further than the smoothing radius, this saves time as unnecessary values are not computed. The influence value is also divided by the volume of the function, giving similar influence values regardless of the smoothing radius.

Below is the implementation of the smoothing kernel derivative used for optimisation discussed in the theoretical model.

\begin{lstlisting}
float smoothingKernelDerivative(double dst){
    if (dst >= smoothing_radius) return 0.0;
    float value = -6.f * (smoothing_radius - dst) * (smoothing_radius - dst) / (pi * pow(smoothing_radius, 4));
    return value;
}
\end{lstlisting}

\subsection{Density Calculations}
Calculating the local densities for each point requires us to use the SPH Interpolation equation, $A_s(r) = \sum_{j} m_j \frac{A_j}{\rho_j}W(r-r_j, h)$ \cite{muller}. Interestingly, to calculate the density at a certian point, substituting for $\rho$ gives us:

\begin{center}
 $ \rho_s(r) = \sum_{j} m_j \frac{\rho_j}{\rho_j}W(r-r_j, h),
 \rho_s(r) = \sum_{j} m_j W(r-r_j, h).$
\end{center}
 Where the densities cancel out, giving a method of calculating the local density without being dependent on it.

Coding this in C++ gives:
\begin{lstlisting}
float calculateDensity(int i){
    float density = 0.f;
    for (int j =0; j < particle_num; j++){
        float dst = sqrt();
        //distance is calculated using predicted positions
        float influence = smoothingKernel(dst);
        density += particle_mass * influence;
    }
    return density;
}
\end{lstlisting}

\subsection{Pressure Calculations}
Quantifying the local density into a pressure value, per the theoretical model, would involve taking the local density and finding a pseudo-pressure value, $P_i = k(\rho_i - \rho_0)$ \cite{clavet}, and applying the gradient optimisation to turn this pseudo-pressure into pressure.

The code below is the implementation of the equation above, where $k$ is the pressure multiplier:
\begin{lstlisting}
float densityToPressure(int j){
    float density_error = particles[j].local_density - target_density;
    float local_pressure = density_error * pressure_multiplier;
    return local_pressure;
}
\end{lstlisting}

Then, in order to apply Newton's 3rd law, the index of the two particles which are being computed is taken and the average of their pseudo-pressures is returned:

\begin{lstlisting}
 float sharedPressure(int i, int j){
    float pressurei = densityToPressure(particles[i]
    .local_density);
    float pressurej = densityToPressure(particles[j]
    .local_density);
    return (-(pressurei + pressurej) / 2.f);
}
\end{lstlisting}
Notice how a negative pressure value is returned, this ensures pressure is a purely repulsive force which makes intuitive sense as the particles must repel each other.

Finally, the pseudo-pressure is converted to a pressure force using the SPH gradient optimisation, which takes into account the rate of change of this property and the $x$ and $y$ direction of the particle to turn the scalar pressure force into a vector. Including the nested loop, the implementation of this is below:

\begin{lstlisting}
 sf::Vector2f calculatePressureForce(int i){
    sf::Vector2f pressure_force;
    sf::Vector2f viscosity_force;
    for (int j = 0; j<particle_num; j++){
        if (i == j) continue;
        float x_offset;
        float y_offset;
        if (particles[i].
        predicted_position == particles[j].
        predicted_position){
            x_offset = (float)1+ (rand() % 20);
            y_offset = (float)1+ (rand() % 20);
        }
        else{
            x_offset = particles[j].predicted_position.x - particles[i].predicted_position.x;
            y_offset = particles[j].predicted_position.y - particles[i].predicted_position.y;
        }
        double dst = sqrt(abs(x_offset * x_offset + y_offset * y_offset));
        float gradient = smoothingKernelDerivative(dst);
        float x_dir = x_offset/dst;
        float y_dir = y_offset/dst;
        float shared_pressure = sharedPressure(i, j);
        pressure_force.x += shared_pressure * gradient * particle_mass
        /particles[j].local_density * x_dir;
        pressure_force.y += shared_pressure * gradient * particle_mass/particles[j]
        .local_density * y_dir;
    }
    return pressure_force;
}
\end{lstlisting}
Notice how a random direction is picked to apply the force if two particles have the same position, this contributes to the particles' Brownian motion.

The returned pressure force must now be converted to an acceleration. This can be done in the main simulation loop by calling the calculatePressureForce function and dividing the value by the density, as per Sebastian Lague's implementation \cite{Lague} of an SPH solver. He mentions this is due to the particles being spheres of volume instead of mass. The implementation within the simulation loop looks like this:

\begin{lstlisting}
sf::Vector2f pressure_force = calculatePressureForce(i);
sf::Vector2f pressure_acceleration;
pressure_acceleration.x = pressure_force.x/particles[i]
.local_density;
pressure_acceleration.y = pressure_force.y/particles[i]
.local_density;
\end{lstlisting}
\subsection{Viscosity}
As calculating viscosity requires a weighted sum from every other particle within the smoothing radius, just like pressure, a nested loop can be used to achieve this. From the theoretical model, implementation in C++ looks like the following:
\begin{lstlisting}

sf::Vector2f calculateViscosityAcceleration(int i){
    sf::Vector2f viscosity_acceleration;
    float dst;
    for (int j; j<particle_num; j++){
        if (i==j) continue;
        float dst;
        //dst is calculated using predicted positions
        viscosity_acceleration.x -= (particles[i].velocity.x - particles[j].velocity.x) * (smoothingKernel(dst)) * viscosity_strength;
        viscosity_acceleration.y -= (particles[i].velocity.y - particles[j].velocity.y) * (smoothingKernel(dst)) * viscosity_strength;
        if (viscosity_acceleration.x > 0 || viscosity_acceleration.y > 0){
            viscosity_acceleration.x = 0;
            viscosity_acceleration.y = 0;
        }
    }
    return viscosity_acceleration;
}

\end{lstlisting}

The acceleration caused by the viscosity within a fluid slows the particle down, this is because viscosity is caused by friction within a fluid, a force which opposes motion. This is why the acceleration caused by viscosity is not added, but rather subtracted from the overall acceleration of the particle.
\subsection{Interactivity and visuals}
One criterion for this project is interactivity. Using SFML, I have allowed the user to resize the window but ImGui will allow the user to use sliders and change key SPH properties discussed in this paper such as the target density, pressure multiplier and the smoothing radius. Additionally, I will also allow for gravity, number of particles and particle mass to also be changed to allow the user freedom and to explore different types of fluid which are possible to simulate using this technique. ImGui interactive variable initialisation code is below:

\begin{lstlisting}
ImGui::Begin("Menu");
ImGui::SliderInt("Particle Num", &particle_num, 1, 1600);
ImGui::SliderFloat("Particle Mass", &particle_mass, 0.1f, 100.f);
ImGui::SliderFloat("Target Density", &target_density, 0.f, 500.f);
ImGui::SliderFloat("Pressure Multiplier", &pressure_multiplier, 0.f, 1000.f);
ImGui::SliderFloat("Smoothing Radius", &smoothing_radius, 10, 200);
ImGui::SliderFloat("Gravity", &gravity, 0.f, 100.f);
ImGui::SliderInt("Framerate", &framerate, 60, 1200);
ImGui::End();
\end{lstlisting}

In terms of the visuals of the project, cyan particles were being rendered earlier. Instead, a more aesthetically pleasing and useful algorithm to implement would be a linear interpolation colour algorithm. Simply put, the algorithm would output a colour for the particle depending on it's velocity. A larger velocity would output a more red-shifted colour whereas a slower velocity would output a blue-shifted colour. In C++, this looks like:

\begin{lstlisting}
 void resolveColor(int i, float vel){
    int b = clamp((int)(-255/50 * vel + 255), 0, 255);
    int r = clamp((int)(255/50 * vel -255), 0, 255);
    int g = clamp((int)(-abs(255/50 * (vel-50))+255), 0, 255);
    particles[i].droplet.setFillColor
    (sf::Color(r, g, b));
}
\end{lstlisting}
This will be run in the simulation loop after the velocities have been calculated for all particles in that frame. Finally, using SFML's in-built support for mouse features, a mouse force function has been coded and is shown below. The program adds the appropriate acceleration within the main simulation loop in the next section.

\begin{lstlisting}
sf::Vector2f interactiveForce(sf::Vector2i position, int i, float repulsive){
    sf::Vector2f force;
    float dst = (particles[i].position.x - position.x) * (particles[i].position.x - position.x) + (particles[i].position.y - position.y) * (particles[i].position.y - position.y);
    if (dst<=10000){
        force += (((sf::Vector2f)position) - particles[i].predicted_position) * (100-sqrt(dst))/4.f * repulsive;
    }
    return force;
}
\end{lstlisting}
The program looks for a left or right click within the events loop.
\begin{lstlisting}
if (event.type == sf::Event::MouseButtonPressed){
    interactive = true;
    //Determine whether it is an attractive or repulsive force
    if (event.mouseButton.button == sf::Mouse::Left){
        repulsive = 1.f;
    }
    else{
        repulsive = -1.f;
    }
}
//Stop the interactive forces is the mouse button is released
if (event.type == sf::Event::MouseButtonReleased){
    interactive = false;
}
\end{lstlisting}

The intention behind this is to be able to better explain fluids moving down the pressre gradient by visualising them, as well as evaluate any anomalies within the simulation.
\subsection{Updating Simulation loop}
Finally, the simulation loop is updated by putting together everything within development so far. This is the updated simulation loop:

\begin{lstlisting}
 for (int i = 0; i < particle_num; i++){
    resolveCollisions(i, window_size);
    //gravity step
    resolveGravity(i);
    //Predict next positions
    predictPositions(i);
    // window bounding box
    //calculate densities
    particles[i].local_density = calculateDensity(i);
    //convert density to pressure
    particles[i].local_pressure = densityToPressure(i);
    //Calculate pressure forces and acceleration
    sf::Vector2f pressure_force = calculatePressureForce(i);
    // Add the mouse forces
    if (interactive){
        pressure_force += interactiveForce(position, i, repulsive);
    }
    sf::Vector2f pressure_acceleration;
    pressure_acceleration.x = pressure_force.x/particles[i].local_density;
    pressure_acceleration.y = pressure_force.y/particles[i].local_density;
    //Calculate acceleration due to viscosity
    pressure_acceleration += calculateViscosityAcceleration(i);
    //calculate velocity
    particles[i].velocity.x += pressure_acceleration.x * dt;
    particles[i].velocity.y += pressure_acceleration.y * dt;
    //resolve colour
    float vel = sqrt(particles[i].velocity.x * particles[i].velocity.x + particles[i].velocity.y * particles[i].velocity.y);
    resolveColour(i, vel);
    //calculate particle positions with radius offset
    particles[i].position.x += particles[i].velocity.x * dt;
    particles[i].position.y += particles[i].velocity.y * dt;
    //set particle position on screen
    particles[i].droplet.setPosition
    (particles[i].position.x+particle_radius, particles[i].position.y+particle_radius);
    //render particle
    window.draw(particles[i].droplet);
}
\end{lstlisting}
\pagebreak
\subsection{Final Product}
The simulation clearly solves each section of the Navier-Stokes equation through SPH. Figure \ref{fig:pressure_grad} in Appendix \ref{appendix:a} shows the fluid moving down the pressure gradient from an area of high pressure at the start of the simulation, to areas of low pressure as the simulation border is extended. By sucessfully attempting to move down this pressure gradient, the fluid constantly attempts to achieve constant density and therefore incompressibility. This is visible in Figure \ref{fig:constant_density} where the simulation eventually reaches a stable state with particles having no/minimal velocities (shown by their blue colour) and arranging themselves in an orderly manner. Viscosity is harder to show but as the particles move from high to low pressure, red particles with unusually high velocities bounce back from the border and match the velocities of the particles around them, turning green and eventually blue as seen in Figure \ref{fig:viscosity}. This also helps the fluid reach a constant density throughout the simulation space.

As for interactivity, the window resizing has worked sucessfully since the start of this project and can be noticed throughout development. Figure \ref{fig:sliders} in Appendix \ref{appendix:a} shows the sliders on screen which control different aspects of the simulation as discussed, available for the user to control the behaviour of the fluid. The ultimate test of interactivity is the mouse forces, which can be seen in Figure \ref{fig:mouse_forces}. Attractive and repulsive forces are working with left and right click respectively with dragging. The particles around them adjust for change in density appropriately.

Refer to Appendix \ref{appendix:b} for the C++ code in its entirety.
\end{document}
