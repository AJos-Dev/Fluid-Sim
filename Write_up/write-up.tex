\documentclass[a4paper,11pt]{article}
\usepackage[utf8]{inputenc}
\usepackage[paper=a4paper]{geometry}
\usepackage[numbers]{natbib}
\usepackage{hyperref}
\hypersetup{
    colorlinks=true,
    linkcolor=darkgray,
    filecolor=magenta,
    urlcolor=cyan,
    citecolor=green
    }
\usepackage{multicol}
\usepackage{sectsty}
\usepackage{subfiles}
\usepackage{enumitem}
\usepackage{physics}
\usepackage{pgfplots}
\usepackage{amsmath}
\pgfplotsset{width=7cm, compat=1.9}
\newlist{worddefs}{description}{1}
\setlist[worddefs]{font=\bfseries, labelindent=\parindent, leftmargin=6em, style=sameline}

%\sectionfont{\fontsize{11}{11}\selectfont}
%\subsectionfont{\fontsize{11}{11}\selectfont}


%opening
\title{\textbf{Simulating Fluid Motion using Smoothed Particle Hydrodynamics}}
\author{Aayush Joshi}
\date{}

\begin{document}
\maketitle

\par\noindent\rule{\textwidth}{0.3pt}

\begin{abstract}
\noindent \textit{Realistic simulation of fluids is an important tool with a wide variety of applications such as within the Aerospace industry to model fluid based phenomena of spacecraft parts and within the computer games industry for authentic graphics. In this paper, I explore a method for simulating fluids known as Smoothed Particle Hydrodynamics (SPH) in order to better understand the mathematical theory behind Computational Fluid Dynamic methods and their implementation in an appropriate programming language, C++.}
%not the way to do an abstract, no intro just state what you cover in each section
\end{abstract}

\par\noindent\rule{\textwidth}{0.3pt}
\tableofcontents
\newpage
\begin{multicols}{2}
\subfile{intro.tex}
\pagebreak
\subfile{research_review.tex}
\pagebreak
\subfile{theoretical_model.tex}
\section{Development and Testing}
\section{Evaluation and Final Remarks}

\newpage
\nocite{*}
\bibliography{bibfile.bib}
\bibliographystyle{IEEEtran}
\end{multicols}
\end{document}


