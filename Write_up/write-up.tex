\documentclass[a4paper,11pt]{article}
\usepackage[utf8]{inputenc}
\usepackage[paper=a4paper]{geometry}
\usepackage[numbers]{natbib}
\usepackage{hyperref}
\hypersetup{
    colorlinks=true,
    linkcolor=black ,
    filecolor=magenta,
    urlcolor=cyan,
    }


%opening
\title{\textbf{Simulating Fluid Motion using Smoothed Particle Hydrodynamics}}
\author{Aayush Joshi}
\date{}

\begin{document}
\maketitle

\par\noindent\rule{\textwidth}{0.3pt}

\begin{abstract}
\noindent \textit{Realistic simulation of fluids is an important tool with a wide variety of applications such as within the Aerospace industry to model fluid based phenomena of spacecraft parts and within the computer games industry for authentic graphics. In this paper, I explore a method for simulating fluids known as Smoothed Particle Hydrodynamics (SPH) in order to better understand the mathematical theory behind Computational Fluid Dynamic methods and their implementation in an appropriate programming language, C++.}
\end{abstract}

\par\noindent\rule{\textwidth}{0.3pt}
\tableofcontents
\newpage

\section{Introduction}
\subsection{Motivation}

\subsection{Smoothed Particle Hydrodynamics}
\subsection{Outline and Structure}
\subsection{Skills and Intentions}

\section{Literature Review}
\section{Theoretical model}
\section{Development and Testing}
\section{Conclusion and Final Remarks}

\newpage
\bibliographystyle{IEEEtranN}
\bibliography{bibfile.bib}
\end{document}
