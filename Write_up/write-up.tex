\documentclass[a4paper,11pt]{article}
\usepackage[utf8]{inputenc}
\usepackage[paper=a4paper]{geometry}
\usepackage[numbers]{natbib}
\usepackage{hyperref}
\hypersetup{
    colorlinks=true,
    linkcolor=black ,
    filecolor=magenta,
    urlcolor=cyan,
    }
\usepackage{multicol}
\usepackage{lipsum}
\usepackage{sectsty}

\sectionfont{\fontsize{13}{13}\selectfont}
\subsectionfont{\fontsize{12}{12}\selectfont}


%opening
\title{\textbf{Simulating Fluid Motion using Smoothed Particle Hydrodynamics}}
\author{Aayush Joshi}
\date{}

\begin{document}
\maketitle

\par\noindent\rule{\textwidth}{0.3pt}

\begin{abstract}
\noindent \textit{Realistic simulation of fluids is an important tool with a wide variety of applications such as within the Aerospace industry to model fluid based phenomena of spacecraft parts and within the computer games industry for authentic graphics. In this paper, I explore a method for simulating fluids known as Smoothed Particle Hydrodynamics (SPH) in order to better understand the mathematical theory behind Computational Fluid Dynamic methods and their implementation in an appropriate programming language, C++.}
\end{abstract}

\par\noindent\rule{\textwidth}{0.3pt}
\tableofcontents
\newpage

\begin{multicols}{2}
\section{Introduction}
\subsection{Motivation}
The field of simulation is one with many applications in all industries, with much overlap between Mathematics, Physics and Computer Science due to its predictable behaviour. One such application is Computational Fluid Dynamics (CFD), or in other words predicting the movement of fluids, which will be the focus for this project.
\\~\\
\indent Simulating fluids involves observation of fluid phenomena such as wind, weather, ocean waves, waves induced by ships or simply pouring a glass of water. Such phenomena may seem extremely trivial at first glance, but in reality involve an extremely deep understanding of physical, mathematical and algorithmic methods, far beyond the A level specification for any related subjects.
\subsection{Smoothed Particle Hydrodynamics}
Smoothed Particle Hydrodynamics (SPH) is a Lagrangian $^1$ approach to simulating fluids. Initially developed for solving astrophysical problems by Lucy \cite{lucy} and by Gingold and Monaghan \cite{gingold}, it was then adapted for interactive liquid simulation and rendering by M{\"u}ller \textit{et al.} \cite{muller}. In SPH, space is non-uniformly sampled using particles. These particles are small globules of a fluid which maintain various field properties such as mass, density or velocity and are tracked during the simulation. The field quantities can be evaluated anywhere in space by looking at the overlaps of each particle's sphere of influence.
\nocite{*}
\subsection{Outline and Structure}
%Incompressible non-gasseous fluid
\subsection{Skills and Intentions}
%Explore a topic not taught at A levels, beyond comfort zone, etc, etc
%Learn a new programming langauge
%Learn to typeset in LaTeX, write an official paper.
%Useful at university
\section{Research Review}
\section{Theoretical model}
\section{Development and Testing}
\section{Conclusion and Final Remarks}

\newpage
\bibliographystyle{IEEEtranN}
\bibliography{bibfile.bib}
\end{multicols}
\end{document}
