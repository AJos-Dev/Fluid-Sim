\documentclass[write-up.tex]{subfiles}
\begin{document}

\section{Theoretical model}
\subsection{Smoothing Kernel}

The fundamental property every SPH particle has is a circle of influence which is dictated by the Smoothing Kernel $W(r, h)$, where $r$ is the distance between the sample point and the particle centre and $h$ is the radius of the circle of influence, also known as the smoothing radius. The closer a sample point is to the centre of the particle, the larger the influence the particle has on the sample point. This influence value can be described and quantified by picking an appropriate smoothing kernel, which is vital as it is used by the \hyperlink{Interpolation Equation}{\textbf{Interpolation Equation}} to calculate scalar quantities, most notably density.

Müller \textit{et al.}\cite{muller} recommend using one of three smoothing kernels. These are:

\begin{center}
$
    W_{\text{poly}}(r, h) = \displaystyle \frac{315}{64 \pi h^9}
    \begin{cases}
        (h^2-r^2)^3 & 0 \leq r \leq h \\
        0 & \text{otherwise},
    \end{cases}
$
\end{center}

\begin{tikzpicture}
\begin{axis}[
    axis lines = left,
    xlabel = \(r\),
    ylabel = \(W_{poly}(r\text{, }1)\)
]
%Below the red parabola is defined
\addplot [
    domain=0:1,
    samples=200,
    color=cyan,
]
{315/(64 * 3.14159) * (1-x^2)^3};
\end{axis}
\end{tikzpicture}

\begin{center}
$
    W_{\text{spiky}}(r, h) = \displaystyle \frac{15}{\pi h^6}
    \begin{cases}
        (h-r)^3 & 0 \leq r \leq h \\
        0 & \text{otherwise},
    \end{cases}
$
\end{center}

\begin{tikzpicture}
\begin{axis}[
    axis lines = left,
    xlabel = \(r\),
    ylabel = \(W_{spiky}(r\text{, }1)\)
]
%Below the red parabola is defined
\addplot [
    domain=0:1,
    samples=200,
    color=cyan,
]
{15/3.14159 * (1-x)^3};
\end{axis}
\end{tikzpicture}

\begin{center}
$
    W_{\text{viscosity}}(r, h) = \displaystyle \frac{15}{2 \pi h^3}
    \begin{cases}
        \displaystyle -\frac{r^3}{2h^3} + \frac{r^2}{h^2} + \frac{h}{2r} -1 & 0 \leq r \leq h \\
        0 & \text{otherwise},
    \end{cases}
$
\end{center}

\begin{tikzpicture}
\begin{axis}[
    axis lines = left,
    xlabel = \(r\),
    ylabel = \(W_{viscosity}(r\text{, }1)\)
]
%Below the red parabola is defined
\addplot [
    domain=0:1,
    samples=100,
    color=cyan,
]
{15/(2 * 3.14159) * (-x^3/2 + x^2 + 1/(2*x) -1)};
\end{axis}
\end{tikzpicture}

\subsection{Interpolation Equation}
\end{document}
