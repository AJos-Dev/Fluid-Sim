\documentclass[write-up.tex]{subfiles}
\begin{document}

\section{Theoretical model}
\subsection{Smoothing Kernel}

The fundamental property every SPH particle has is a circle of influence which is dictated by the Smoothing Kernel $W(r, h)$, where $r$ is the distance between the sample point and the particle centre and $h$ is the radius of the circle of influence, also known as the smoothing radius. The closer a sample point is to the centre of the particle, the larger the influence the particle has on the sample point. This influence value can be described and quantified by picking an appropriate smoothing kernel, which is vital as it is used by the \hyperlink{Interpolation Equation}{\textbf{Interpolation Equation}} to calculate scalar quantities, most notably density.

Müller \textit{et al.}\cite{muller} describe two popular smoothing kernels, both with different properties.

\begin{center}
$
    W_{\text{poly}}(r, h) = \displaystyle \frac{315}{64 \pi h^9}
    \begin{cases}
        (h^2-r^2)^3 & 0 \leq r \leq h \\
        0 & \text{otherwise},
    \end{cases}
$
\end{center}

\begin{tikzpicture}
\begin{axis}[
    axis lines = left,
    xlabel = \(r\),
    ylabel = \(W_{poly}(r\text{, }1)\)
]
%Below the red parabola is defined
\addplot [
    domain=0:1,
    samples=200,
    color=cyan,
]
{315/(64 * 3.14159) * (1-x^2)^3};
\end{axis}
\end{tikzpicture}

$W_{\text{poly}}(r, h)$ was created as an all purpose kernel with its major highlight being that $r$ appears squared, so the square root does not need to be evaluated when using the Pythagorean Theorem to calculate distance, easing the distance computations. Müller \textit{et al.}\cite{muller} mention that if this kernel is used for pressure computations, which this project will entail to enforce incompressibility, a problem arises where the particles will tend to cluster because the gradient is used to compute pressure. The gradient of $W_{\text{poly}}(r, h)$ drops close to $0$ for small $r$ leading to a diminishing repulsion force.

\begin{center}
$
    W_{\text{spiky}}(r, h) = \displaystyle \frac{15}{\pi h^6}
    \begin{cases}
        (h-r)^3 & 0 \leq r \leq h \\
        0 & \text{otherwise},
    \end{cases}
$
\end{center}

\begin{tikzpicture}
\begin{axis}[
    axis lines = left,
    xlabel = \(r\),
    ylabel = \(W_{spiky}(r\text{, }1)\)
]
%Below the red parabola is defined
\addplot [
    domain=0:1,
    samples=200,
    color=cyan,
]
{15/3.14159 * (1-x)^3};
\end{axis}
\end{tikzpicture}

$W_{\text{spiky}}(r, h)$ is used specifically for pressure computations. Its gradient is high when $r$ is close to $0$, generating the required repulsion forces for pressure calculations and therefore making it the superior choice for my simulation.

\subsection{Interpolation Equation}



\end{document}
