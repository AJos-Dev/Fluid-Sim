\documentclass[write-up.tex]{subfiles}
\begin{document}

\section{Theoretical model}
\subsection{Smoothing Kernel}

Each SPH particle has a circle of influence determined by the Smoothing Kernel $W(r,h)$ where $r$ is the distance from the sample point to the particle center and $h$ is the smoothing radius. The influence of a particle on a sample point increases as the sample point gets closer to the particle center. The choice of smoothing kernel is crucial as it is used by the \hyperlink{Interpolation Equation}{\textbf{Interpolation Equation}} to calculate scalar quantities, such as density. Müller \textit{et al.}\cite{muller} describe two popular smoothing kernels, each with different properties.

\begin{center}
$
    W_{\text{poly}}(r, h) = \displaystyle \frac{315}{64 \pi h^9}
    \begin{cases}
        (h^2-r^2)^3 & 0 \leq r \leq h \\
        0 & \text{otherwise},
    \end{cases}
$
\end{center}

\begin{tikzpicture}
\begin{axis}[
    axis lines = left,
    xlabel = \(r\),
    ylabel = \(W_{poly}(r\text{, }1)\)
]
%Below the red parabola is defined
\addplot [
    domain=0:1,
    samples=200,
    color=cyan,
]
{315/(64 * 3.14159) * (1-x^2)^3};
\end{axis}
\end{tikzpicture}

$W_{\text{poly}}(r, h)$ is a versatile kernel designed to simplify distance computations by squaring the distance term, eliminating the need for square root calculations when using the Pythagorean Theorem. However, Müller \textit{et al.}\cite{muller} note that when $W_{\text{poly}}(r, h)$ is used for pressure computations, as required for enforcing incompressibility in this project, particles tend to cluster due to the gradient's effect on pressure calculation. The gradient of $W_{\text{poly}}(r, h)$ approaches zero for small distances, resulting in a diminishing repulsion force.

\begin{center}
$
    W_{\text{spiky}}(r, h) = \displaystyle \frac{15}{\pi h^6}
    \begin{cases}
        (h-r)^3 & 0 \leq r \leq h \\
        0 & \text{otherwise},
    \end{cases}
$
\end{center}

\begin{tikzpicture}
\begin{axis}[
    axis lines = left,
    xlabel = \(r\),
    ylabel = \(W_{spiky}(r\text{, }1)\)
]
%Below the red parabola is defined
\addplot [
    domain=0:1,
    samples=200,
    color=cyan,
]
{15/3.14159 * (1-x)^3};
\end{axis}
\end{tikzpicture}

$W_{\text{spiky}}(r, h)$ is used specifically for pressure computations. Its gradient is high when $r$ is close to $0$, generating the required repulsion forces for pressure calculations and therefore making it the superior choice for my simulation. However, the $h^6$ term is computationally challenging to calculate for every particle, therefore a more appropriate smoothing kernel for our uses would be a variation of $W_{\text{spiky}}(r, h)$ where $W(r, h) = (h-r)^3$ if $0\leq r \leq h$, 0 otherwise.

For the simulation, the influence value returned will also be divided by the volume of our kernel, this keeps values consistent regardless of our values of $h$.

\subsection{Interpolation Equation}
\hypertarget{Interpolation Equation}
The Interpolation Equation is responsible for calculating a scalar quantity $A$ at a location $r$ by a weighted sum of contributions from all the particles within our simulation. It sits at the heart of this project. This equation will be used to calculate density and viscosity which will indicate the pressure and therefore the net force a particle observes.
\begin{center}
$
   A_s(r) = \sum_{j} m_j \frac{A_j}{\rho_j}W(r-r_j, h)
$

$A_s$ is the property to be calculated,\\
$m_j$ is the mass of the particle,\\
$A_J$ is the value of that property of particle with index $j$,\\
$\rho_j$ is the local density, \\
$W(r-r_j, h)$ is the value of the smoothing kernel with the distance between the 2 particles being $r-r_j$.
\cite{muller}
\end{center}

\subsection{SPH gradient optimisation}

The simulation step updates the position vectors of the particles according to the rate of change of the properties every frame. This rate of change can be calculated by taking the derivative of the smoothing kernel with respect to r:

\begin{align*}
W(r, h) &= (h-r)^3 \\
\pdv{W}{r} &= -3(h-r)^2
\end{align*}

\subsection{Pressure Forces and Newton's Third Law}

Clavet \textit{et al.} \cite{clavet} mention calculating a pseudo-pressure $P_i$, proportional to the difference between the local density $\rho_i$ and the current density $\rho_0$, governed by the equation $P_i = k(\rho_i - \rho_0)$, where $k$ is a constant, which will be referred to as the pressure multipler, governing the stiffness of the fluid. The SPH gradient optimisation is then applied to get a value for the pressure force. By Newton's second law, $F = ma$, the acceleration of the particle can be found and can be appled to the particles within the simulation step. Furthermore, Newton's Third Law of Motion must also be applied in this context, where if a particle exerts a pressure force, it must experience an equal and opposite reaction force.

\subsection{Viscosity}
The nature of this project would lead you to assume that viscosity is calculated using the interpolation equation. However, Koschier \textit{et al.}\cite{koschier} mention there are major issues with this approach. The most significant issue is that this approach is sensitive to particle disorder. An proposed alternative which is viable for the scope of this project is to use ``one derivative using SPH and the second one using finite differences'' \cite{koschier}. Following from Sebastian Lague's implementation, the viscosity force is calculated by taking the difference in velocities, multiplying by the influence from a viscosity kernel and then multiplying by an arbitary scalar value.

For my implementation, I will settle for the smoothing kernel for viscosity for simplicity instead of the suggested sharper viscosity kernel. Mentioned earlier, larger powers are time consuming to compute and therefore it is more time and space efficient to apply the computed smoothing kernel influence value than to calculate values from two seperate kernels with large powers in the same loop.

\subsection{Predicted Position optimisation}
An interesting paper by B. Solenthaler and R. Pajarola \cite{solenthaler} uses the current velocities of particles to determine a predicted position and uses predicted positions in the density and pressure calculations instead of current particle positions. This Predictive-Corrective Incompressible SPH (PCISPH) model allows for greater enforcement on incompressibility whilst having low computational cost per update with a large timestep, which is useful as other methods of enforcing incompressibility rely on smaller timesteps that are computationally heavy.
\end{document}
